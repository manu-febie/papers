\documentclass[12pt]{article}

\usepackage{apacite}
\usepackage{geometry}
\usepackage{fontspec}

\setlength{\parindent}{4em}
\setlenght{\pasrskip}}1em
\renewcommand{\baselinestretch}{1.5}

\geometry{margin=1in}
\setmainfont{Times New Roman}

\begin{document}

\title{Gender Identity Issues-}
\author{Imanuel Febie}
\maketitle

% Introduction
% Hook -Your hook should draw the reader’s interest immediately. Questions are a common way of getting interest, as well as evocative language or a strong statistic
It is estimated that the number of LGBT (lesbian, gay, bisexual, and transgender) adults in the us are a total of 11,343,000, of which 1,397,150 are transger.
 
% Background
% Don’t assume that your audience is already familiar with your topic. Give them some background information, such as a brief history of the issue or some additional context.

 
% Thesis
% Your thesis is the crux of your argument. In an argumentative essay, your thesis should be clearly outlined so that readers know exactly what point you’ll be making. Don’t explain all your evidence in the opening, but do take a strong stance and make it clear what you’ll be discussing.

 
% Body
% Claims -Your claims are the ideas you’ll use to support your thesis. For example, if you’re writing about how your neighborhood shouldn’t use weed killer, your claim might be that it’s bad for the environment. But you can’t just say that on its own—you need evidence to support it.

 
% Evidence
% Evidence is the backbone of your argument. This can be things you glean from scientific studies, newspaper articles, or your own research. You might cite a study that says that weed killer has an adverse effect on bees, or a newspaper article that discusses how one town eliminated weed killer and saw an increase in water quality. These kinds of hard evidence support your point with demonstrable facts, strengthening your argument.

 
% Opposition
% In your essay, you want to think about how the opposition would respond to your claims and respond to them. Don’t pick the weakest arguments, either—figure out what other people are saying and respond to those arguments with clearly reasoned arguments.

% Demonstrating that you not only understand the opposition’s point, but that your argument is strong enough to withstand it, is one of the key pieces to a successful argumentative essay.

 
% Conclusion
% Conclusions are a place to clearly restate your original point, because doing so will remind readers exactly what you’re arguing and show them how well you’ve argued that point.

% Summarize your main claims by restating them, though you don’t need to bring up the evidence again. This helps remind readers of everything you’ve said throughout the essay.

% End by suggesting a picture of a world in which your argument and action are ignored. This increases the impact of your argument and leaves a lasting impression on the reader.

\end{document}
