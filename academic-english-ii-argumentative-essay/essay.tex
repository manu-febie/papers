\documentclass[12pt]{article}

\usepackage{apacite}
\usepackage{geometry}
\usepackage{fontspec}

\setlength{\parindent}{4em}
\setlength{\parskip}{1em}
\renewcommand{\baselinestretch}{1.5}

\setmainfont{Times New Roman}
\geometry{margin=1in}

\begin{document}

\title{Debunking Myths of Sex and Gender Ideology}
\author{Imanuel Febie}
\maketitle

% MAIN POINT OR CLAIM: Biology only allows 2 sexes or genders
The term 'theybies' is adopted by some parents in the United States who have chosen to raise their children gender neutral, with the objective to let the children decide them self what gender they want to identify them self with \cite{raisingTheybies}. These are people who identify them self as non-binary. Currently they are getting higher profile in popular culture, and multiple U.S. states and cities are granting gender-neutral classifications on either the driver-license or birth certificate \cite{nonbinaryGenderAwareness}. Those who identify as nonbinary do not consider them self exclusively male or female, but can be a combination of male and female, or even shifting between male and female \cite{nonBinaryGender2017}. Sex researcher, Debrah Soh, points out that a trend of academic research job postings put more emphasis on whether a candidate belongs to the population being studied, rather than their skills as a researcher \cite{endOfGender2020Soh}. The ideology of a small percentage of people is being enforced while the actual science of the differences between the sexes and gender are being ignored or rather bended to fit the narrative of the non-binary community

Contrary to what present-day popular culture would like to have society believe, is the fact that sex is binary, and not a spectrum. Therefore sex has two available possibilities. For humans this is represented as either male or female. Likewise, masculine and feminine behavior is dictated by biology \cite{personalityAndGenderDifferences}. Typically gender behavior is not the result of a social construct. Confusion often befall on the terms gender and sex, considering they are both related to each other and distinct \cite{psychosexualDifferentiation}. Certainly individuals may express them self, and those who identify them self as nonbinary have the right to express their gender. However, an ideology can not change biological traits, nor does the way an individual might feel about them self, in view of the fact that sex and gender typical traits is already established before birth. Let us first establish the biological rules for sex determination. 

Generally people believe that sex is determined by our chromosomes or our genitals or hormonal profiles \cite{endOfGender2020Soh}. Although not completely incorrect, sex is determined by two distinct reproductive cells called gametes, also know as sex cells. The spermatozoan gamete is produced by the male, and contains either the X or the Y sex chromosome, whereas the female gamete produces the so called ovum only containing the X sex chromosome. When these two gametes unite in a process called fertilization, they develop into what is called zygote which contains two sets of chromosomes. The zygote XX will result in a female while the zygote XY will result into a male. The Y sex chromosome carries the blueprint for the development of male testes, ergo the male gamete is responsible for the determining the sex of the baby when it unites with the female gamete \cite{endOfGender2020Soh}. Therefore, at fertilization the sex of the baby is already determined by the gametes. Certainly there are exceptions to rule, including intersex people, but rules are not made for exceptions.

In addition, structurally the male and female brain do not differ, but functionally they indeed differ from one another \cite{maleVSFemaleBrain2016Glezerman}. Not because one is better or worse, neither more nor less sophisticated, but just different \cite{maleVSFemaleBrain2016Glezerman}, considering the fact that the male brain is exposed to an entirely different hormonal environment inside the womb \cite{maleVSFemaleBrain2016Glezerman}. High exposure to testosterone in the male fetus results in differences, such as behavioral characteristics, toy preferences and many other attributes \cite{maleVSFemaleBrain2016Glezerman}. Typically, males have better motor and spatial capabilities, while females have shown to have superior memory and social cognition skills \cite{sexDifferences2014Ingalhalikar}. Furthermore, for women there is a strong possibility to show interest in social and artistic activities, while men will typically gravitate more towards scientific, technical, and mechanical activities based on a meta-analysis of studies involving a total of over 500,000 respondents \cite{menAndThings2009Sy}. However, this is not absolute and overlaps are not unusual \cite{maleVSFemaleBrain2016Glezerman}. Besides, studies examining countries which are more egalitarian, have shown that women are underrepresented in science, technology, engineering, and mathematics (STEM), creating a paradox \cite{genderParadox}. 

It is commonly believed that gender is constructed socially and culturally \cite{irshad2012gender}. This theory is implying that certain behaviors that are generally accepted to be masculine or feminine to be instructed upon us by society. This idea gained its' prestige through works from feminists like Simone de Beauvoir and Judith Butler \cite{endOfGender2020Soh}. Both argued that women are not born, but made with the believe that gender is a performance. However, the previous studies mentioned in this essay, have established that certain behaviors gravitate either to male or female \cite{menAndThings2009Sy}. Studies also have proved that introducing prenatal testosterone influence both neural and behavioral sexual distinction \cite{prenatalTestos}. Gender is the result of biology not because of environmental influences. This does however allow for differences individually, and in most cases male and female characteristics overlap each other to some degree. 
 
        Science can not and should not be ignored in matters like sex and gender. The process of sex determination only allows the possibility of two genders, either male or female, with some rare exceptions. Exposure to testosterone inside the womb or lack of it, will dictate typical behavior that are considered to be male or female. Certainly, masculine or feminine traits is defined by culture, however whether an individual will lean more towards either one is dictated by biology \cite{endOfGender2020Soh}. The battle should not to be erasing gender but our behavior towards the other sex. Men and women are different in behavior, but they fulfill each other. To conclude, this essay also showed that the more egalitarian a country is, women actually are underrepresented in STEM fields, proving that gender is not something socially constructed but driven by biology. We can not change what sex and gender already is according to science.

\bibliographystyle{apacite}
\bibliography{../ref.bib}

\end{document}
