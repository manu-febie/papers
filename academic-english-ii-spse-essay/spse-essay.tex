\documentclass[12pt]{article}

% note: 800 words minimum, 1000 words max

\usepackage{apacite}
\usepackage{geometry}
\usepackage{fontspec}

\setmainfont{Times New Roman}
\geometry{margin=1in}

\begin{document}

\title{The Significance of Active Learning Classrooms to Increase Student Engagement}
\author{Imanuel Febie, 2201835800}

\maketitle

% introduction - what is the problem and its effects
The purpose of a lecture in an academic environment is to provide students with information or ideas, whether this delivery method is conducted in the traditional form or online. It has been the prevalent form of teaching since universities were founded in Western Europe in 1050 \cite{bajak2014lectures}. An academic lecture can be good for presenting information along with explanation, but has often been criticized because it lacks interaction with the listener. Thus not being an effective learning method \cite{what-students-want}. Efforts to change and improve traditional learning into alternative classrooms like the Active Learning Classroom (ALC), have emerged since the mid-1990s \cite{guide-to-teaching-alc}. Active learning has shown that it can increase the students' enthusiasm, thus having a positive effect on the students' academic achievements. This essay will present two possible solutions in the realm of active learning environments - that is, learning by teaching and the flipped classroom - because universities could improve their students academic achievements.

% Problem description
Research has shown that 59\% of students find their lectures, dull half of the time and 30\% find most if not all of their lectures to be dull \cite{boredom-among-students}. To cope with boring lectures, students indulge in activities like daydreaming, chatting with friends or simply just leaving the classroom \cite{predictors-of-boredom}. Lecturers should not be focusing on only pouring knowledge into a students' brain, but they should encourage students into learning \cite{satisfaction-active-learning}. The lecturer is challenged to gain and maintain the student interest. A study has shown that students are 1.5 times more likely to fail with the traditional stand-and-deliver lectures \cite{freeman-active-learning-2014}. An important issue in this matter is the lack of engagement from the students' perspective. Boredom among students in higher education is not foreign and has been extensively studied and has been connected to numerous adverse consequences such as decreased academic achievement, dissatisfaction and truancy \cite{boredom-among-students}. 

% solution one - learning by teaching
The engagement with the learning material is an important factor. One method that has proved to be successful is by having students teach. This method of teaching is often also labeled as 'students as partners' and 'co-creation in learning'. In short, students are being allowed to prepare and teach lessons to their fellow students. According to Cook-Sather et al (2014), it is both important for the student and the facilitator that the partnership has its foundations build on respect, reciprocity, and shared responsibility between student and faculty \cite{engaging-students}. Do not confuse this learning environment with a student representative, which often is reserved for a smaller group of elected students representing a larger group of students. Instead students are truly engaged in the teaching process by accepting roles like the representative, consultant, co-researcher, and pedagogical co-designer. Instead of just passively sitting and listening to a lecture presenting power point presentation, the student now shares a responsibility for what takes place in the learning environment and is also able to make more sense of what they are learning together with their lecturers \cite{cocreation-in-learning}. Gathered from a range of sources of evidence, \cite{cocreation-in-learning} we can see the positive outcomes. One student states, "Enhanced identity, meta cognitive awareness of learning and teaching, inspired, and/or transformed \cite{investigation-of-cocreation}". Where often the grades are the primary focus, a student now understand the focus should be on learning \cite{voices-in-the-study-of-teaching-and-learning}. Teaching forces the student to become familiar with the content, but universities can also decide to flip things around to keep students engaged.

% solution two
Traditionally, the teaching approach has been lecturing a class and activities are given as homework. Another approach is to make the lecture content available online for study outside of class and in-class sessions are used to deepen the understanding of the content. This is commonly known as the "flipped classroom" and are thought to be student-centred and also implies that students are at the heart of the learning experience \cite{understanding-flipped-classroom}. The role of the lecturer still remains important as that of a facilitator for the students to motivate, guide, and give feedback on their performance \cite{flip-your-classroom}. As a result of this approach, students' have more time to solve problems individually or collaboratively, in contrast to listening to long lectures in the classroom. Studies also have proved the positive outcome of the flipped classroom on increased attendance and academic performance \cite{the-flipped-classroom}. As with learning by teaching, the student does not have a passive role in class anymore, but is required to know the basics before class. The classroom is used for learning activities like problem-solving or but not limited to case-based discussions to reinforce the knowledge they already had constructed on their own. Repeated exposure to information as is the case in the flipped classroom, will lead to storing information in the long term memory \cite{impact-of-flipped-classroom}. The hypothesis that students enrolled in flipped classrooms would perform better in their final exams, have higher quiz scores, and rating courses higher was tested by Nielsen et al (2018), turned out to be correct. In their study, students performed better and had a positive view towards the course \cite{impact-of-flipped-classroom}.

% evaluation - how effective are the solutions?
%            - which is most effective
Active learning has proved to be much more effective the traditional lectures.Not only academic performance is improved, but also the perspective of the student towards learning in general. Students learn the value of learning. This essay discussed how two active learning environments can be used to engage the student by letting the student become a partner of the faculty or by flipping the traditional classroom by requiring the student to have the basic understanding  of the content pre-class. However, both environments can be approached differently by different universities. Although the initial start of those new to active learning can be experienced as frustrating, the effort pays off, and the majority agrees that they learned more \cite{guide-to-teaching-alc}. There is evidence that active classrooms produce better performing students. This indicates that universities are in need for restructuring.

\newpage

% Bibliography
\bibliographystyle{apacite}
\bibliography{../ref.bib}

\end{document}
