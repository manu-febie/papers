\documentclass[12pt]{article}

% note: 800 words minimum, 1000 words max

\usepackage{apacite}
\usepackage{geometry}
\usepackage{fontspec}

\setmainfont{Times New Roman}
\geometry{margin=1in}

\begin{document}

\title{Boredom Among Students and Gaining Interest with Active Learning Classrooms}
\author{Imanuel Febie, 2201835800}
\maketitle

% introduction - what is the problem and its effects
The purpose of a lecture in an academic environment is to provide students with information or ideas, whether this delivery method is conducted in the traditional form or online. Lecturing has been the prevalent form of teaching since universities were founded in Western Europe in 1050 \cite{bajak2014lectures}. However, a study has shown that students are 1.5 times more likely to fail with the traditional stand-and-deliver lectures \cite{freeman-active-learning-2014}. One of the reasons is because of boring lectures. Boredom among students in higher education is not foreign and has been extensively studied and has been connected to numerous adverse consequences such as decreased academic achievement, dissatisfaction and truancy \cite{boredom-among-students}. Lecturers should not be focusing on only pouring knowledge into a students brain, but they should encourage students into learning \cite{satisfaction-active-learning}. The lecturer is challenged to gain and maintain the student interest. Efforts to change and improve traditional learning into alternative classrooms have emerged since the mid-1990s \cite{guide-to-teaching-alc}. This essay will discuss the concern regarding traditional lectures and two possible solutions utilizing Active Learning Classrooms.

% solution one - active learning 
One research has shown that 59\% of students find their lectures dull half of the time and 30\% find most if not all of their lectures to be dull \cite{boredom-among-students}. In order to cope with boring lectures, students indulge into activities like daydreaming, chatting with friends or simply just leaving the classroom \cite{predictors-of-boredom}. A student has to be engaged with the teaching in order to keep being focused. In order to improve teaching methods, universities introduced active-learning environments. 

% solution two


% evaluation - how effective are the solutions?
%            - which is most effective

\newpage

% Bibliography
\bibliographystyle{apacite}
\bibliography{../ref.bib}

\end{document}
