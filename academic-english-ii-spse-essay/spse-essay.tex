\documentclass[12pt]{article}

% note: 800 words minimum, 1000 words max

\usepackage{apacite}
\usepackage{geometry}
\usepackage{fontspec}

\setmainfont{Times New Roman}
\geometry{margin=1in}

\begin{document}

\title{Significance of Active Learning Classrooms to Increase Student Engagement}
\author{Imanuel Febie, 2201835800}
\maketitle

% introduction - what is the problem and its effects
The purpose of a lecture in an academic environment is to provide students with information or ideas, whether this delivery method is conducted in the traditional form or online. Lecturing has been the prevalent form of teaching since universities were founded in Western Europe in 1050 \cite{bajak2014lectures}. However, a study has shown that students are 1.5 times more likely to fail with the traditional stand-and-deliver lectures \cite{freeman-active-learning-2014}. One of the reasons is because of boring lectures. Boredom among students in higher education is not foreign and has been extensively studied and has been connected to numerous adverse consequences such as decreased academic achievement, dissatisfaction and truancy \cite{boredom-among-students}. Lecturers should not be focusing on only pouring knowledge into a students brain, but they should encourage students into learning \cite{satisfaction-active-learning}. The lecturer is challenged to gain and maintain the student interest. Efforts to change and improve traditional learning into alternative classrooms like the Active Learning Classroom (ALC), have emerged since the mid-1990s \cite{guide-to-teaching-alc}. Active learning has shown that it can increase the students enthousiasm, thus having a positive effect on a students academic achievements. This essay will present two possible solutions in the realm of active learning environments - that is, learning by teaching and the flipped classroom - because universities could improve their students academic achievements.

% solution one - learning by teaching
One research has shown that 59\% of students find their lectures dull half of the time and 30\% find most if not all of their lectures to be dull \cite{boredom-among-students}. In order to cope with boring lectures, students indulge into activities like daydreaming, chatting with friends or simply just leaving the classroom \cite{predictors-of-boredom}. A student has to be engaged with the teaching in order to keep being focused. One method that has proved to be successful is by having students teach. This method of teaching is often also labeled as 'students as partners' and 'co-creation in learning'. In short, students are being allowed to prepare and teach lessons to their fellow students. According to Cook-Sather et al (2014), it is both important for the student and the facilitator that the partnership has its foundations build on respect, reciprocity, and shared responsibility between student and faculty \cite{engaging-students}. Do not confuse this learning environment with a student representative which often is reserved for a smaller group of elected students representing a larger group of students. Instead student are truly engaged in the teaching process by accepting roles like: representative, consultant, co-researcher, and pedagogical co-designer. Instead of just passively sitting and listening to a lecture presenting power point presentation, the student now shares a responsibility for what takes place in the learning environment and is also able to make more sense of what they are learning together with their lecturers \cite{cocreation-in-learning}. 

% solution two


% evaluation - how effective are the solutions?
%            - which is most effective

\newpage

% Bibliography
\bibliographystyle{apacite}
\bibliography{../ref.bib}

\end{document}
